%                                                                 
% AA vers. 9.1, LaTeX class for Astronomy & Astrophysics - modified to 9.2
%
%                                                       (c) EDP Sciences
%-----------------------------------------------------------------------
%
%\documentclass[referee]{aa} % for a referee version
%\documentclass[onecolumn]{aa} % for a paper on 1 column  
%\documentclass[longauth]{aa} % for the long lists of affiliations 
%\documentclass[letter]{aa} % for the letters 
%\documentclass[bibyear]{aa} % if the references are not structured 
%                              according to the author-year natbib style

%
\documentclass[a4paper]{aa} % a&a template, on a4 because we're european thank you very much

\usepackage{natbib, twoopt} % a&a uses this
\bibpunct{(}{)}{;}{a}{}{,} % to follow the A&A style
%
\usepackage{amsmath} % standard for math
%\usepackage{mathptmx} % change math font (?, might be doing nothing)
\usepackage{siunitx} % SI units support
\usepackage{graphicx} % make figures
\usepackage{subcaption} % make subfigures
\usepackage{lipsum} % lorem ipsum, a&a examples were using this
\usepackage{url} % make url links
\usepackage[hyphenbreaks]{breakurl} % allow breaking urls
\usepackage{xcolor} % make colour names available (used for links)
\definecolor{cobalt}{rgb}{0.06, 0.2, 0.65} % new color
% a&a example stuff
\usepackage{lscape}             % to rotate a single page table, example in appendix.
                                % For landscape tables, see the longtable examples.
\usepackage{placeins}           % useful with \FloatBarrier, to keep 
                                % onecolumn floats from drifting to the next section
%%%%%%%%%%%%%%%%%%%%%%%%%%%%%%%%%%%%%%%%
%\usepackage{txfonts}
\usepackage[varg]{txfonts} % change general font as a&a recommends
%%%%%%%%%%%%%%%%%%%%%%%%%%%%%%%%%%%%%%%%
\usepackage[colorlinks, breaklinks]{hyperref} % linking inside pdf, with custom colours and allow wrapping links
% orange numbers for sections etc, citations in red, urls in blue
\hypersetup{%
  colorlinks = true,
  linkcolor  = orange,
  citecolor = red,
  urlcolor = blue,
  pdftitle={Cosmology II Project}
}
% To add links in your PDF file, use the package "hyperref"
% with options according to your LaTeX or PDFLaTeX drivers.
%

% remove warnings about empty links when using hyperref
% https://tex.stackexchange.com/questions/345764/journal-class-shows-package-hyperref-warning-suppressing-link-with-empty-targe
% and add hyperlinks to citations
% https://github.com/yangcht/AA-bibstyle-with-hyperlink
\makeatletter
% fix the page number
\renewcommand*\aa@pageof{, page \thepage{} of \pageref*{LastPage}}
% change some other displayed information, this is not a formal A&A paper after all
\renewcommand*\aa@numarticle{Cosmology II numerical project} % changing article "number" since this isn't published, don't mind the hack
% display in top box at the beginning
\renewcommand*\aa@manuscriptname{%
  templated numerical project
  \hspace{\stretch{1}}%
  \copyright UiO \the\year
}
% top text on even pages
\renewcommand*\aa@textidlineempty{{\slshape A\&A template:}\ Cosmology II}

% required for the hyperlinks, see https://github.com/yangcht/AA-bibstyle-with-hyperlink
\newcommandtwoopt{\citeads}[3][][]{\href{http://adsabs.harvard.edu/abs/#3}%
    {\def\hyper@linkstart##1##2{}%
     \let\hyper@linkend\@empty\citealp[#1][#2]{#3}}}
\newcommandtwoopt{\citepads}[3][][]{\href{http://adsabs.harvard.edu/abs/#3}%
    {\def\hyper@linkstart##1##2{}%
     \let\hyper@linkend\@empty\citep[#1][#2]{#3}}}
\newcommandtwoopt{\citetads}[3][][]{\href{http://adsabs.harvard.edu/abs/#3}%
    {\def\hyper@linkstart##1##2{}%
     \let\hyper@linkend\@empty\citet[#1][#2]{#3}}}
\newcommandtwoopt{\citeyearads}[3][][]%
    {\href{http://adsabs.harvard.edu/abs/#3}
    {\def\hyper@linkstart##1##2{}%
     \let\hyper@linkend\@empty\citeyear[#1][#2]{#3}}}

% custom hack to make eprint command have an actual arxiv link
\renewcommand*\eprint[2][]{%
{\tt\if!#1!#2\else{\href{https://arxiv.org/abs/#2}{\ignorespaces#1\arxivprefixesep\ignorespaces#2}}\fi}%
}
\makeatother


\begin{document} 


   \title{Make a suitable title: my paper on the cosmic microwave background and formation of structures in our Universe}
   \titlerunning{Cosmology II Project}

   \subtitle{AST5220: Cosmology II}

   \author{Candidate 15003
          %\inst{1}
          }

   \institute{Institute of Theoretical Astrophysics,  
   University of Oslo,  0315 Oslo,  Norway\\
             }

   \date{Received \today}
   %\date{Received September 15, 1996; accepted March 16, 1997}

%-------------------------------------------------------------------
%-------------------------------------------------------------------
% \abstract{}{}{}{}{} 
% 5 {} token are mandatory
 
  \abstract
  % context heading (optional)
  % {} leave it empty if necessary  
   {Something about a CAMB solver}
  % aims heading (mandatory)
   {Cosmological simulations}
  % methods heading (mandatory)
   {Calculations are done}
  % results heading (mandatory)
   {Results are resulted}
  % conclusions heading (optional), leave it empty if necessary 
   {An abstract for the paper. Describe the paper. What is the paper about, what are the main results, etc.}

   \keywords{cosmic microwave background --
                large-scale structure of Universe --
                %structure formation --
                recombination
               }

   \maketitle
%
%-------------------------------------------------------------------

%-------------------------------------------------------------------
\section{Introduction}

Write an introduction here. 
Give context to the paper. 
Citations to relevant papers. 
You only need to do this in the end for the last milestone.
   
%-------------------------------------------------------------------

\section{Milestone I}\label{sec:milestone_1}
Some introduction about what it is all about.

Cite: \citet{baumannLectureNotesCosmology2017} or \citet{dodelsonModernCosmology2021} \citep{callinHowCalculateCMB2006, wintherCosmologyIILecture2024, huCompleteTreatmentCMB1998}

\subsection{Theory}
The theory behind this milestone. See Friedmann equation \ref{eq:Friedmann}

Fiducial cosmology and initial parameter values taken from Planck 2018 results \citep{collaborationPlanck2018Results2020}.

\begin{equation}\label{eq:Friedmann}
    \boxed{H = H_0 \sqrt{ \Omega_{M 0} a^{-3} + \Omega_{R 0} a^{-4} + \Omega_{k 0} a^{-2} + \Omega_{\Lambda 0}}},
\end{equation}

where the $\Omega_{X}$ are density parameters describing relative density of their respective form of energy contributing to the expansion of the universe. Subscript $_0$ indicates a value for the universe of today, since we as observers are by definition at $x=a=z=0$.

$\Omega_{M 0} = (\Omega_{b0}+\Omega_{\rm CDM 0})$ is a composite density parameter describing non-relativistic matter, and $\Omega_{R 0} = (\Omega_{\gamma 0} + \Omega_{\nu 0})$ is a composite density term for radiation.

Paper with supernova fitting data \citet{betouleImprovedCosmologicalConstraints2014}

Equation for critical density of the universe today \ref{eq:critical_density}

\begin{equation}\label{eq:critical_density}
\rho_{c0} \equiv \frac{3H_0^2}{8\pi G}
\end{equation}

\subsection{Implementation details}
Something about the numerical work.

\subsection{Results}
Show and discuss the results.

See figs. \ref{fig:milestone_1_cosmic_vs_conformal_time}, \ref{fig:milestone_1_luminosity_distance}, \ref{fig:milestone_1_etaHp_over_c_of_x}, \ref{fig:milestone_1_Omega_i_of_x}, \ref{fig:milestone_1_supernovafitting_confidence_regions}.

\begin{figure}
\centering
\includegraphics[width=0.4\textwidth]{../Milestone 1/Plots/cosmic_vs_conformal_time.png}
\caption{Caption}
\label{fig:milestone_1_cosmic_vs_conformal_time}
\end{figure}

\begin{figure}
\centering
\includegraphics[width=0.4\textwidth]{../Milestone 1/Plots/luminosity_distance.png}
\caption{Caption}
\label{fig:milestone_1_luminosity_distance}
\end{figure}

\begin{figure}
\centering
\includegraphics[width=0.4\textwidth]{../Milestone 1/Plots/etaHp_over_c_of_x.png}
\caption{Caption}
\label{fig:milestone_1_etaHp_over_c_of_x}
\end{figure}

\begin{figure}
\centering
\includegraphics[width=0.4\textwidth]{../Milestone 1/Plots/Omega_i_of_x.png}
\caption{Caption}
\label{fig:milestone_1_Omega_i_of_x}
\end{figure}

\begin{figure}
\centering
\includegraphics[width=0.4\textwidth]{../Milestone 1/Plots/supernovafitting_confidence_regions.png}
\caption{Caption}
\label{fig:milestone_1_supernovafitting_confidence_regions}
\end{figure}

\section{Milestone II}\label{sec:milestone_2}
With fundamental cosmology established in sec. \ref{sec:milestone_1}, we can now describe the baseline or so-called "background" behaviour of our Universe, with a relatively simple evolution of cosmological paremeters as the universe expands (refer fig. \ref{fig:milestone_1_Omega_i_of_x}). Now we wish to look backwards from our current time, and compute the path of photons travelling towards a current-day observer from the early universe.

In order to study the behaviour of photons and thermal evolution of the early universe, we consider it to be a large continuous fluid, specifically a hot plasma. The thermodynamics and statistical mechanics for this are described in \citet[][chap.~x]{baumannLectureNotesCosmology2017} and \citet[][chap.~x]{dodelsonModernCosmology2003}, while the specific Boltzmann formalism utilized is that of \citet[][]{wintherCosmologyIILecture2024}, \href{https://cmb.wintherscoming.no/theory_thermodynamics.php#thermo}{which can be found here}.

\subsection{Theory}
The theory behind this milestone.

Start with eqs. \ref{eq:tau_integral} and \ref{eq:tau_ODE}.

\begin{equation}\label{eq:tau_integral}
\tau(\eta) = \int_{\eta}^{\eta_0} n_e \sigma_T a d\eta'
\end{equation}

\begin{equation}\label{eq:tau_ODE}
\boxed{\tau' = \frac{d\tau}{dx} = -\frac{c n_e \sigma_T }{H}.}
\end{equation}

We wish to compute the fractional electron density given by \ref{eq:fractional_electron_density}, where we assume all baryons are protons and there are no heavier elements. This approximation is acceptable for getting a simple reionization with clear falloff of electrons as they get absorbed into hydrogen atoms. By including ionization into Helium, our resulting ionization plot would have multiple bumps for ionization into different states of Hydrogen+Helium.


\begin{equation}\label{eq:fractional_electron_density}
\boxed{X_e \equiv n_e / n_H}, \text{ with } \,\, n_H = n_b \approx \frac{\rho_b}{m_H} = \frac{\Omega_{b0} \rho_{c0}}{m_H a^3}
\end{equation}


Saha approximation \ref{eq:saha_approx}

\begin{equation}\label{eq:saha_approx}
\boxed{\frac{X_e^2}{1-X_e} = \frac{1}{n_b} \left(\frac{m_e
T_b}{2\pi}\right)^{3/2} e^{-\epsilon_0/T_b}}
\end{equation}

Peebles equation \ref{eq:peebles_equation}, with the supporting definitions in eq. \ref{eq:peebles_quantities_definition}.

\begin{equation}\label{eq:peebles_equation}
\boxed{\frac{dX_e}{dx} = \frac{C_r(T_b)}{H} \left[\beta(T_b)(1-X_e) - n_H
\alpha^{(2)}(T_b)X_e^2\right],}
\end{equation}

\subsection{Implementation details}
Something about the numerical work.

\subsection{Results}
Show and discuss the results.

%\FloatBarrier
\section{Milestone III}
Now we introduce density perturbations to our baseline cosmology, and study the evolution of these perturbations. Perturbations apply to photon multipoles, baryon and dark matter density and velocity, and to the gravitational metric.

Citations: \citet{maCosmologicalPerturbationTheory1995}

\subsection{Theory}
We can derive a system of ODEs to solve in order to study perturbations. See \citet[part 3, cosmological perturbation theory]{wintherCosmologyIILecture2024} for derivation of these equations. The main equations to solve are eqs. \ref{eq:cdm_baryon_perturbations} and \ref{eq:metric_perturbations}, for matter perturbations (CDM, baryons) and perturbations of the graviational metric respectively.

\begin{equation}\label{eq:cdm_baryon_perturbations}
\boxed{
\begin{aligned}
\delta_{\rm CDM}^\prime &= \frac{ck}{\mathcal{H}} v_{\rm CDM} - 3\Phi^\prime \\
v_{\rm CDM}^\prime &= -v_{\rm CDM} -\frac{ck}{\mathcal{H}} \Psi \\
\delta_b^\prime &= \frac{ck}{\mathcal{H}}v_b -3\Phi^\prime \\
v_b^\prime &= -v_b - \frac{ck}{\mathcal{H}}\Psi + \tau^\prime R(3\Theta_1 + v_b) \\
\end{aligned}
}
\end{equation}

\begin{equation}\label{eq:metric_perturbations}
\boxed{
\begin{aligned}
\Phi^\prime &= \Psi - \frac{c^2k^2}{3\mathcal{H}^2} \Phi + \\
    &+ \frac{H_0^2}{2\mathcal{H}^2} \left[\Omega_{\rm CDM 0} a^{-1} \delta_{\rm CDM} + \Omega_{b 0} a^{-1} \delta_b + 4\Omega_{\gamma 0}a^{-2}\Theta_0\right] \\
\Psi &= -\Phi - \frac{12H_0^2}{c^2k^2a^2}\left[\Omega_{\gamma 0}\Theta_2\right] \\
\end{aligned}
}
\end{equation}

$R$ is the same as earlier, $R = \frac{4\Omega_{\gamma 0}}{3\Omega_{b 0} a}$.

However, these equations are affected by the temperature and photon hotspots in the fluid we are considering to be the universe. Photons, unlike baryons, have multipolar effects, which enter as $\Theta_\ell$. Each $\Theta_\ell$ depends on $\Theta_{\ell+1}$, which makes for an infinite set of equations to solve. Thankfully, line of sight integration as developed by (Zaldarriaga and Seljak) can solve this problem, providing an alternate means of calculating a power spectrum from the Boltzmann equations we started with. (See lecture notes citation). Line of sight integration only needs a few starting multipoles for verification, so in this project we will truncate the multipole hierarchy at around six.

Apart from photon multipoles, there are also additional effects from both neutrino multipoles and polarization of the photon fluid. These will not be treated in this project, instead see for example (something where they actually do it). But here, we treat $\mathcal{N}_\ell = 0$ and $\Theta^P_\ell = 0$.

The equations for photon multipoles to be included are eq. \ref{eq:photon_multipoles}.

\begin{equation}\label{eq:photon_multipoles}
\boxed{
\begin{aligned}
\Theta^\prime_0 &= -\frac{ck}{\mathcal{H}} \Theta_1 - \Phi^\prime \\
\Theta^\prime_1 &=  \frac{ck}{3\mathcal{H}} \Theta_0 - \frac{2ck}{3\mathcal{H}}\Theta_2
    + \frac{ck}{3\mathcal{H}}\Psi + \tau^\prime\left[\Theta_1 + \frac{1}{3}v_b\right] \\
&\begin{split}
\Theta^\prime_\ell = \frac{\ell ck}{(2\ell+1)\mathcal{H}}&\Theta_{\ell-1} - \frac{(\ell+1)ck}{(2\ell+1)\mathcal{H}} \Theta_{\ell+1} \\
    &+ \tau^\prime\left[\Theta_\ell - \frac{1}{10}\Pi \delta_{\ell,2}\right], \quad 2 \le \ell < \ell_{\textrm{max}}
    \end{split} \\
\Theta_{\ell}^\prime &= \frac{ck}{\mathcal{H}} \Theta_{\ell-1}
    - c\frac{\ell+1}{\mathcal{H}\eta(x)}\Theta_\ell + \tau^\prime\Theta_\ell, \quad \ell = \ell_{\textrm{max}}\\
\end{aligned}
}
\end{equation}

\subsection{Implementation details}
%Something about the numerical work.

\subsection{Results}
See figs. \ref{fig:milestone_3_delta_gamma_delta_b_delta_cdm}, \ref{fig:milestone_3_v_b_v_cdm}, \ref{fig:milestone_3_phi}, \ref{fig:milestone_3_phi_plus_psi}, \ref{fig:milestone_3_theta_0}, \ref{fig:milestone_3_theta_1}.

Recombination happened too late, and whatever was going on there seems to have severely interfered with these results. No time to investigate properly, though. Presumably the pipeline is fine and just needs fixes to the calculations, and then these plots would turn out as they should.

\begin{figure}[h!tbp]
\centering
\includegraphics[width=0.4\textwidth]{../Milestone 3/Plots/delta_gamma_delta_b_delta_cdm_plot.png}
\caption{Perturbations to densities, represented by the parameter $\delta$ presented by \citet{wintherCosmologyIILecture2024}. CDM perturbations are represented by the solid line, baryon perturbations by the dashed line, and photon perturbations by the dotted line. Photon perturbations evolve mostly unaffected by the matter perturbations, which is good. Baryons might be oscillating away from the dark matter (due to baryon dragging) as expected, but fall off the scale of the plot. Early time development is completely messed up. Different scales might be entering evolution at different times as expected, but this is hard to tell.}
\label{fig:milestone_3_delta_gamma_delta_b_delta_cdm}
\end{figure}

\begin{figure}[h!tbp]
\centering
\includegraphics[width=0.4\textwidth]{../Milestone 3/Plots/v_b_v_cdm_plot.png}
\caption{Similar to fig. \ref{fig:milestone_3_delta_gamma_delta_b_delta_cdm}, but for perturbed average velocities of the respective quantities. Early times are completely messed up.}
\label{fig:milestone_3_v_b_v_cdm}
\end{figure}

\begin{figure}[h!tbp]
\centering
\includegraphics[width=0.4\textwidth]{../Milestone 3/Plots/phi_plot.png}
\caption{Perturbations to the graviational metric, component $\Phi$. The scale dependency is wrong, modes that enter early should fall off to zero much more aggressively. Perturbations should start around the same size and stay relatively constant until re-entering the horizon. We do not observe the change in evolution as the universe starts to become dominated by dark energy at the very end.}
\label{fig:milestone_3_phi}
\end{figure}

\begin{figure}[h!tbp]
\centering
\includegraphics[width=0.4\textwidth]{../Milestone 3/Plots/phi_plus_psi_plot.png}
\caption{Combined perturbations to the graviational metric. Same problems as fig. \ref{fig:milestone_3_phi}.}
\label{fig:milestone_3_phi_plus_psi}
\end{figure}

\begin{figure}[h!tbp]
\centering
\includegraphics[width=0.4\textwidth]{../Milestone 3/Plots/theta_0_plot.png}
\caption{The evolution of the photon monopole. Modes should be re-entering at different times to start a significant dampened oscillation, but the plot is messed up.}
\label{fig:milestone_3_theta_0}
\end{figure}

\begin{figure}[h!tbp]
\centering
\includegraphics[width=0.4\textwidth]{../Milestone 3/Plots/theta_1_plot.png}
\caption{Evolution of the photon dipole. Same problems as fig. \ref{fig:milestone_3_theta_0}.}
\label{fig:milestone_3_theta_1}
\end{figure}

%\section{Milestone IV}
I got completely stuck on milestone III. I was looking forward to III and finally getting to some of the physics there, too

Citations: \citet{fixsenCosmicMicrowaveBackground1996}

\subsection{Theory}
CMB and matter power spectrums, how they work and the different regimes in them. How a sum of coefficients describes the different scale components of a sky-map. Line of sight integration to calculate all the various $\Theta_\ell$-s.

How the different regimes in the power spectrum plots are affected by different shifts in cosmological parameters. Cosmic variance in the beginning at low $\ell$-s. $A_s$ for overall amplitude of the whole spectrum, $n_s$ describes tilt. The fluctuation peaks after the main peak tell us about baryon and dark matter density. Also affected by polarization, neturinos, and reionization which is why the simpler power spectrum we would have derived would be off from the Planck spectrum in this regime, while the lower $\ell$-s match fine.

The matter power spectrum tells us about the total matter component in the universe, but the interesting part is the second small peak and subsequent dampened fluctuations that tell us about the ratio of baryons to dark matter, due to the effect of baryon dragging.

The source function:

\begin{equation}\label{eq:source_function}
\begin{aligned}
    \tilde{S}(k,x) = \tilde{g}\left[ \Theta_0 + \Psi + \frac{1}{4}\Pi\right] + e^{-\tau} \left[\Psi^\prime-\Phi^\prime\right] - \\
    - \frac{1}{ck}\frac{d}{dx}(\mathcal{H}\tilde{g}v_b) + \frac{3}{4c^2k^2} \frac{d}{dx} \left[\mathcal{H}\frac{d}{dx} (\mathcal{H}\tilde{g}\Pi)\right].
\end{aligned}
\end{equation}

This describes various effects in the universe that have affected photons travelling from the surface of last scattering (the CMB photons) to us, today. The baseline movement is in the first term (hence the dependency on visibility function). The term with $\psi$ dependence is a correction due to the Sachs-Wolfe effect, the term with $\Pi$ is a correction for the effect of the photon quadropole and some polarization. The next main term is a correction for the integrated Sachs-Wolfe effect (ISW), which should have one of those neat diagrams to show how photons gain energy due to gravitational wells decaying while they are climbing out. Next up is a correction for Doppler shift, and finally a correction for angular dependence in Thompson scattering.

\subsection{Implementation details, results}
It did not work

\section{Conclusions}

   \begin{enumerate}
      \item The conditions for the stability of static, radiative
         layers in gas spheres, as described by Baker's (\citeyear{baker})
         standard one-zone model, can be expressed as stability
         equations of state. These stability equations of state depend
         only on the local thermodynamic state of the layer.
      \item If the constitutive relations -- equations of state and
         Rosseland mean opacities -- are specified, the stability
         equations of state can be evaluated without specifying
         properties of the layer.
      \item For solar composition gas the $\kappa$-mechanism is
         working in the regions of the ice and dust features
         in the opacities, the $\mathrm{H}_2$ dissociation and the
         combined H, first He ionization zone, as
         indicated by vibrational instability. These regions
         of instability are much larger in extent and degree of
         instability than the second He ionization zone
         that drives the Cephe{\"\i}d pulsations.
   \end{enumerate}

\begin{acknowledgements}
      Part of this work was supported by the German
      \emph{Deut\-sche For\-schungs\-ge\-mein\-schaft, DFG\/} project
      number Ts~17/2--1.
\end{acknowledgements}
%-------------------------------------------------------------------


%-------------------------------------------------------------------
% - use BibTeX with the regular commands:
%   \bibliographystyle{aa} % style aa.bst
%   \bibliography{Yourfile} % your references Yourfile.bib
%
% - join the .bib files when you upload your source files

\bibliographystyle{bibtex/aa_url}
\bibliography{bibtex/bibliography.bib}

%-------------------------------------------------------------------

\begin{appendix}

\FloatBarrier
\section{Code repository}
All code used for this project is available at \href{https://github.com/ericludvigs/AST5220_Cosmology_Project}{this Github repository}.
The src folder contains the numerical C++ code. Calculated results are in csv-files found under the results folder. Presentation is handled by Python scripts per milestone, in the respective Milestone X folders. Produced plots are saved to a Plots folder under each Milestone.

\FloatBarrier
\section{Milestone I, extra plots}\label{app:milestone_1_extra_plots}

Evolution of some physical quantities in figs. \ref{fig:milestone_1_H_prime_of_x} and \ref{fig:milestone_1_eta_of_x}.

\begin{figure}[h!tb]
\centering
\includegraphics[width=0.4\textwidth]{../Milestone 1/Plots/H_prime_of_x.png}
\caption{Direct plot of $\mathcal{H}$. For reference the current accepted value of $h$ is plotted, which intercepts the plot as expected at $x=0$. We see the expected steady falloff with a slight change in steepness as the universe becomes matter dominated, and a reversal in the size of $H$ as the universe becomes dominated by dark energy.}
\label{fig:milestone_1_H_prime_of_x}
\end{figure}

\begin{figure}[h!bt]
\centering
\includegraphics[width=0.4\textwidth]{../Milestone 1/Plots/eta_of_x.png}
\caption{Direct plot of the conformal time. Relates to cosmic horizons, which expand slowly in the radiation dominated era.}
\label{fig:milestone_1_eta_of_x}
\end{figure}

Histograms of parameter distribution in fig. \ref{fig:milestone_1_appendix_histograms}.

\begin{figure*}[h!tb]
\centering
    \begin{subfigure}[t!]{0.4\textwidth}
    \centering
    \includegraphics[width=1.0\textwidth]{../Milestone 1/Plots/H0_histogram.png}
    \caption{Accepted samples for $H_0$.}
    \label{fig:milestone_1_H0_histogram}
    \end{subfigure}
    %\hfill
    \begin{subfigure}[t!]{0.4\textwidth}
    \centering
    \includegraphics[width=1.0\textwidth]{../Milestone 1/Plots/OmegaK_histogram.png}
    \caption{Accepted samples for $\Omega_K$.}
    \label{fig:milestone_1_OmegaK_histogram}
    \end{subfigure}
    \hfill
    %\hfill
    \begin{subfigure}[b!]{0.4\textwidth}
    \centering
    \includegraphics[width=1.0\textwidth]{../Milestone 1/Plots/OmegaM_histogram.png}
    \caption{Accepted samples for $\Omega_M$.}
    \label{fig:milestone_1_OmegaM_histogram}
    \end{subfigure}
    \begin{subfigure}[b!]{0.4\textwidth}
    \centering
    \includegraphics[width=1.0\textwidth]{../Milestone 1/Plots/OmegaLambda_histogram.png}
    \caption{Accepted samples for $\Omega_\Lambda$.}
    \label{fig:milestone_1_OmegaLambda_histogram}
    \end{subfigure}
\caption{Histogram of accepted samples from the MCMC, with a simple gaussian function with the same mean and standard deviation overplotted. Sampling covers a suitably random range and should be a good representation of possible parameter values. The final best-fit values are indicated.}
\label{fig:milestone_1_appendix_histograms}
\end{figure*}

%\FloatBarrier
\section{Milestone II, extra math}
Definitions to support Peebles equation (\ref{eq:peebles_equation}):

\boxed{
\begin{aligned}\label{eq:peebles_quantities_definition}
C_r(T_b) &= \frac{\Lambda_{2s\rightarrow1s} + \Lambda_{\alpha}}{\Lambda_{2s\rightarrow1s} + \Lambda_{\alpha} + \beta^{(2)}(T_b)}, \enspace \text{[dimensionless]}\\
H &,  \enspace \text{[\unit{1/s}]}\\
\Lambda_{2s\rightarrow1s} &= 8.227, \enspace \text{[\unit{1/s}]}\\
\Lambda_{\alpha} &= H\frac{(3\epsilon_0)^3}{(8\pi)^2 c^3 \hbar^3 n_{1s}}, \enspace \text{[\unit{1/s}]}\\
n_{1s} &= (1-X_e)n_H, \enspace \text{[\unit{1/m^3}]}\\
n_H &= (1-Y_p) n_b \approx (1-Y_p) \frac{3H_0^2\Omega_{b0}}{8\pi G m_H a^3}, \enspace \text{[\unit{1/m^3}]}\\
\beta^{(2)}(T_b) &= \beta(T_b) e^{\frac{3\epsilon_0}{4 k_b T_b}}, \enspace \text{[\unit{1/s}]}\\
\beta(T_b) &= \alpha^{(2)}(T_b) \left(\frac{m_e k_b T_b}{2\pi \mathbf{\hbar}^2}\right)^{3/2} e^{-\frac{\epsilon_0}{k_b T_b}}, \enspace \text{[\unit{1/s}]} \\
\alpha^{(2)}(T_b) &= \frac{8}{\sqrt{3\pi}} c \sigma_T \sqrt{\frac{\epsilon_0}{k_b T_b}}\phi_2(T_b), \enspace \text{[\unit{m^3/s}]}\\
\phi_2(T_b) &= 0.448\ln\left(\frac{\epsilon_0}{k_b T_b}\right), \enspace \text{[dimensionless]}\\
\sigma_T &= \frac{8\pi}{3} \left(\frac{\alpha \hbar c}{m_e c^2}\right)^2, \enspace \text{[\unit{m^2}]}\\
    &\begin{aligned}
    \alpha \simeq \frac{1}{137.0359992} \enspace &\text{[dimensionless,}\\
    &\text{{fine-structure constant}]}
    \end{aligned}
\end{aligned}
}

\FloatBarrier
\section{Milestone III, extra math}

Initial conditions for \ref{eq:milestone_3_ode_initial_conditions}:

\begin{equation}\label{eq:milestone_3_ode_initial_conditions}
\boxed{
\begin{aligned}
\Psi &= -\frac{1}{\frac{3}{2} + \frac{2f_\nu}{5}}\\
\Phi &= -(1+\frac{2f_\nu}{5})\Psi \\
\delta_{\rm CDM} &= \delta_b = -\frac{3}{2} \Psi \\
v_{\rm CDM} &= v_b = -\frac{ck}{2\mathcal{H}} \Psi\\
&\text{Photon multipoles:}\\
\Theta_0 &= -\frac{1}{2} \Psi \\
\Theta_1 &= +\frac{ck}{6\mathcal{H}}\Psi \\
\Theta_2 &= -\frac{20ck}{45\mathcal{H}\tau^\prime} \Theta_1 \quad\quad \textrm{(without polarization)} \\
\Theta_\ell &= -\frac{\ell}{2\ell+1} \frac{ck}{\mathcal{H}\tau^\prime} \Theta_{\ell-1}\\
\end{aligned}
}
\end{equation}

\end{appendix}


%-------------------------------------------------------------------
\end{document}
