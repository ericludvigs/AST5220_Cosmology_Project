\section{Baker's standard one-zone model}

%-------------------------------------- Two column figure (place early!)
   \begin{figure*}
   \centering
   %%%\includegraphics{empty.eps}
   %%%\includegraphics{empty.eps}
   %%%\includegraphics{empty.eps}
   \caption{Adiabatic exponent $\Gamma_1$.
               $\Gamma_1$ is plotted as a function of
               $\lg$ internal energy $\mathrm{[erg\,g^{-1}]}$ and $\lg$
               density $\mathrm{[g\,cm^{-3}]}$.}
              \label{FigGam}%
    \end{figure*}
%
   In this section the one-zone model of \citet{baker},
   originally used to study the Cephe{\"{\i}}d pulsation mechanism, will
   be briefly reviewed. The resulting stability criteria will be
   rewritten in terms of local state variables, local timescales and
   constitutive relations.

   \citet{baker} investigates the stability of thin layers in
   self-gravitating,
   spherical gas clouds with the following properties:
   \begin{itemize}
      \item hydrostatic equilibrium,
      \item thermal equilibrium,
      \item energy transport by grey radiation diffusion.
   \end{itemize}
   For the one-zone-model Baker obtains necessary conditions
   for dynamical, secular and vibrational (or pulsational)
   stability (Eqs.\ (34a,\,b,\,c) in Baker \citeyear{baker}). Using Baker's
   notation:
   %%begin novalidate
   \[
      \begin{array}{lp{0.8\linewidth}}
         M_{r}  & mass internal to the radius $r$     \\
         m               & mass of the zone                    \\
         r_0             & unperturbed zone radius             \\
         \rho_0          & unperturbed density in the zone     \\
         T_0             & unperturbed temperature in the zone \\
         L_{r0}          & unperturbed luminosity              \\
         E_{\mathrm{th}} & thermal energy of the zone
      \end{array}
   \]
   %%end novalidate
\noindent
   and with the definitions of the \emph{local cooling time\/}
   (see Fig.~\ref{FigGam})
   \begin{equation}
      \tau_{\mathrm{co}} = \frac{E_{\mathrm{th}}}{L_{r0}} \,,
   \end{equation}
   and the \emph{local free-fall time}
   \begin{equation}
      \tau_{\mathrm{ff}} =
         \sqrt{ \frac{3 \pi}{32 G} \frac{4\pi r_0^3}{3 M_{\mathrm{r}}}
}\,,
   \end{equation}
   Baker's $K$ and $\sigma_0$ have the following form:
   \begin{eqnarray}
      \sigma_0 & = & \frac{\pi}{\sqrt{8}}
                     \frac{1}{ \tau_{\mathrm{ff}}} \\
      K        & = & \frac{\sqrt{32}}{\pi} \frac{1}{\delta}
                        \frac{ \tau_{\mathrm{ff}} }
                             { \tau_{\mathrm{co}} }\,;
   \end{eqnarray}
   where $ E_{\mathrm{th}} \approx m (P_0/{\rho_0})$ has been used and
   \begin{equation}
   \begin{array}{l}
      \delta = - \left(
                    \frac{ \partial \ln \rho }{ \partial \ln T }
                 \right)_P \\
      e=mc^2
   \end{array}
   \end{equation}
   is a thermodynamical quantity which is of order $1$ and equal to $1$
   for nonreacting mixtures of classical perfect gases. The physical
   meaning of $ \sigma_0 $ and $K$ is clearly visible in the equations
   above. $\sigma_0$ represents a frequency of the order one per
   free-fall time. $K$ is proportional to the ratio of the free-fall
   time and the cooling time. Substituting into Baker's criteria, using
   thermodynamic identities and definitions of thermodynamic quantities,
   \begin{displaymath}
      \Gamma_1      = \left( \frac{ \partial \ln P}{ \partial\ln \rho}
                           \right)_{S}    \, , \;
      \chi^{}_\rho  = \left( \frac{ \partial \ln P}{ \partial\ln \rho}
                           \right)_{T}    \, , \;
      \kappa^{}_{P} = \left( \frac{ \partial \ln \kappa}{ \partial\ln P}
                           \right)_{T}
   \end{displaymath}
   \begin{displaymath}
      \nabla_{\mathrm{ad}} = \left( \frac{ \partial \ln T}
                             { \partial\ln P} \right)_{S} \, , \;
      \chi^{}_T       = \left( \frac{ \partial \ln P}
                             { \partial\ln T} \right)_{\rho} \, , \;
      \kappa^{}_{T}   = \left( \frac{ \partial \ln \kappa}
                             { \partial\ln T} \right)_{T}
   \end{displaymath}
   one obtains, after some pages of algebra, the conditions for
   \emph{stability\/} given
   below:
   \begin{eqnarray}
      \frac{\pi^2}{8} \frac{1}{\tau_{\mathrm{ff}}^2}
                ( 3 \Gamma_1 - 4 )
         & > & 0 \label{ZSDynSta} \\
      \frac{\pi^2}{\tau_{\mathrm{co}}
                   \tau_{\mathrm{ff}}^2}
                   \Gamma_1 \nabla_{\mathrm{ad}}
                   \left[ \frac{ 1- 3/4 \chi^{}_\rho }{ \chi^{}_T }
                          ( \kappa^{}_T - 4 )
                        + \kappa^{}_P + 1
                   \right]
        & > & 0 \label{ZSSecSta} \\
     \frac{\pi^2}{4} \frac{3}{\tau_{ \mathrm{co} }
                              \tau_{ \mathrm{ff} }^2
                             }
         \Gamma_1^2 \, \nabla_{\mathrm{ad}} \left[
                                   4 \nabla_{\mathrm{ad}}
                                   - ( \nabla_{\mathrm{ad}} \kappa^{}_T
                                     + \kappa^{}_P
                                     )
                                   - \frac{4}{3 \Gamma_1}
                                \right]
        & > & 0   \label{ZSVibSta}
   \end{eqnarray}
%
   For a physical discussion of the stability criteria see \citet{baker} or \citet{cox}.

   We observe that these criteria for dynamical, secular and
   vibrational stability, respectively, can be factorized into
   \begin{enumerate}
      \item a factor containing local timescales only,
      \item a factor containing only constitutive relations and
         their derivatives.
   \end{enumerate}
   The first factors, depending on only timescales, are positive
   by definition. The signs of the left hand sides of the
   inequalities~(\ref{ZSDynSta}), (\ref{ZSSecSta}) and (\ref{ZSVibSta})
   therefore depend exclusively on the second factors containing
   the constitutive relations. Since they depend only
   on state variables, the stability criteria themselves are \emph{
   functions of the thermodynamic state in the local zone}. The
   one-zone stability can therefore be determined
   from a simple equation of state, given for example, as a function
   of density and
   temperature. Once the microphysics, i.e.\ the thermodynamics
   and opacities (see Table~\ref{KapSou}), are specified (in practice
   by specifying a chemical composition) the one-zone stability can
   be inferred if the thermodynamic state is specified.
   The zone -- or in
   other words the layer -- will be stable or unstable in
   whatever object it is imbedded as long as it satisfies the
   one-zone-model assumptions. Only the specific growth rates
   (depending upon the time scales) will be different for layers
   in different objects.

%--------------------------------------------------- One column table
   \begin{table}
      \caption[]{Opacity sources.}
         \label{KapSou}
     $$ 
         \begin{array}{p{0.5\linewidth}l}
            \hline
            \noalign{\smallskip}
            Source      &  T / {[\mathrm{K}]} \\
            \noalign{\smallskip}
            \hline
            \noalign{\smallskip}
            Yorke 1979, Yorke 1980a & \leq 1700^{\mathrm{a}}     \\
%           Yorke 1979, Yorke 1980a & \leq 1700             \\
            Kr\"ugel 1971           & 1700 \leq T \leq 5000 \\
            Cox \& Stewart 1969     & 5000 \leq             \\
            \noalign{\smallskip}
            \hline
         \end{array}
     $$ 
   \end{table}
%
   We will now write down the sign (and therefore stability)
   determining parts of the left-hand sides of the inequalities
   (\ref{ZSDynSta}), (\ref{ZSSecSta}) and (\ref{ZSVibSta}) and thereby
   obtain \emph{stability equations of state}.

   The sign determining part of inequality~(\ref{ZSDynSta}) is
   $3\Gamma_1 - 4$ and it reduces to the
   criterion for dynamical stability
   \begin{equation}
     \Gamma_1 > \frac{4}{3}\,\cdot
   \end{equation}
   Stability of the thermodynamical equilibrium demands
   \begin{equation}
      \chi^{}_\rho > 0, \;\;  c_v > 0\, ,
   \end{equation}
   and
   \begin{equation}
      \chi^{}_T > 0
   \end{equation}
   holds for a wide range of physical situations.
   With
   \begin{eqnarray}
      \Gamma_3 - 1 = \frac{P}{\rho T} \frac{\chi^{}_T}{c_v}&>&0\\
      \Gamma_1     = \chi_\rho^{} + \chi_T^{} (\Gamma_3 -1)&>&0\\
      \nabla_{\mathrm{ad}}  = \frac{\Gamma_3 - 1}{\Gamma_1}         &>&0
   \end{eqnarray}
   we find the sign determining terms in inequalities~(\ref{ZSSecSta})
   and (\ref{ZSVibSta}) respectively and obtain the following form
   of the criteria for dynamical, secular and vibrational
   \emph{stability}, respectively:
   \begin{eqnarray}
      3 \Gamma_1 - 4 =: S_{\mathrm{dyn}}      > & 0 & \label{DynSta}  \\
%
      \frac{ 1- 3/4 \chi^{}_\rho }{ \chi^{}_T } ( \kappa^{}_T - 4 )
         + \kappa^{}_P + 1 =: S_{\mathrm{sec}} > & 0 & \label{SecSta} \\
%
      4 \nabla_{\mathrm{ad}} - (\nabla_{\mathrm{ad}} \kappa^{}_T
                             + \kappa^{}_P)
                             - \frac{4}{3 \Gamma_1} =: S_{\mathrm{vib}}
                                      > & 0\,.& \label{VibSta}
   \end{eqnarray}
   The constitutive relations are to be evaluated for the
   unperturbed thermodynamic state (say $(\rho_0, T_0)$) of the zone.
   We see that the one-zone stability of the layer depends only on
   the constitutive relations $\Gamma_1$,
   $\nabla_{\mathrm{ad}}$, $\chi_T^{},\,\chi_\rho^{}$,
   $\kappa_P^{},\,\kappa_T^{}$.
   These depend only on the unperturbed
   thermodynamical state of the layer. Therefore the above relations
   define the one-zone-stability equations of state
   $S_{\mathrm{dyn}},\,S_{\mathrm{sec}}$
   and $S_{\mathrm{vib}}$. See Fig.~\ref{FigVibStab} for a picture of
   $S_{\mathrm{vib}}$. Regions of secular instability are
   listed in Table~1.

%
%                                                One column figure
%----------------------------------------------------------------- 
   \begin{figure} \label{FigVibStab}
   \centering
   \includegraphics[width=3cm]{figures/empty.eps}
      \caption{Vibrational stability equation of state
               $S_{\mathrm{vib}}(\lg e, \lg \rho)$.
               $>0$ means vibrational stability.
              }
   \end{figure}
%-----------------------------------------------------------------
