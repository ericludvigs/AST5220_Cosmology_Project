%-------------------------------------------------------------------
% \abstract{}{}{}{}{} 
% 5 {} token are mandatory
 
  \abstract
  % context heading (optional)
  % {} leave it empty if necessary  
   {To investigate the physical nature of the `nuc\-leated instability' of
   proto giant planets, the stability of layers
   in static, radiative gas spheres is analysed on the basis of Baker's
   standard one-zone model.}
  % aims heading (mandatory)
   {It is shown that stability
   depends only upon the equations of state, the opacities and the local
   thermodynamic state in the layer. Stability and instability can
   therefore be expressed in the form of stability equations of state
   which are universal for a given composition.}
  % methods heading (mandatory)
   {The stability equations of state are
   calculated for solar composition and are displayed in the domain
   $-14 \leq \lg \rho / \mathrm{[g\, cm^{-3}]} \leq 0 $,
   $ 8.8 \leq \lg e / \mathrm{[erg\, g^{-1}]} \leq 17.7$. These displays
   may be
   used to determine the one-zone stability of layers in stellar
   or planetary structure models by directly reading off the value of
   the stability equations for the thermodynamic state of these layers,
   specified
   by state quantities as density $\rho$, temperature $T$ or
   specific internal energy $e$.
   Regions of instability in the $(\rho,e)$-plane are described
   and related to the underlying microphysical processes.}
  % results heading (mandatory)
   {Vibrational instability is found to be a common phenomenon
   at temperatures lower than the second He ionisation
   zone. The $\kappa$-mechanism is widespread under `cool'
   conditions.}
  % conclusions heading (optional), leave it empty if necessary 
   {}

   \keywords{giant planet formation --
                $\kappa$-mechanism --
                stability of gas spheres
               }

   \maketitle
%
%-------------------------------------------------------------------
