\section{Introduction}

   In the \emph{nucleated instability\/} (also called core
   instability) hypothesis of giant planet
   formation, a critical mass for static core  envelope
   protoplanets has been found. \citet{mizuno} determined
   the critical mass of the core to be about $12 \,M_\oplus$
   ($M_\oplus=5.975 \times 10^{27}\,\mathrm{g}$ is the Earth mass), which
   is independent of the outer boundary
   conditions and therefore independent of the location in the
   solar nebula. This critical value for the core mass corresponds
   closely to the cores of today's giant planets.

   Although no hydrodynamical study has been available many workers
   conjectured that a collapse or rapid contraction will ensue
   after accumulating the critical mass. The main motivation for
   this article
   is to investigate the stability of the static envelope at the
   critical mass. With this aim the local, linear stability of static
   radiative gas  spheres is investigated on the basis of Baker's
   (\citeyear{baker}) standard one-zone model. 

   Phenomena similar to the ones described above for giant planet
   formation have been found in hydrodynamical models concerning
   star formation where protostellar cores explode
   (Tscharnuter \citeyear{tscharnuter}, Balluch \citeyear{balluch}),
   whereas earlier studies found quasi-steady collapse flows. The
   similarities in the (micro)physics, i.e., constitutive relations of
   protostellar cores and protogiant planets serve as a further
   motivation for this study.
   
%-------------------------------------------------------------------
