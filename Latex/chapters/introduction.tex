\section{Introduction}

This paper seeks to study the fundamental cosmology of our Universe and its early evolution, from right after inflation to the present day at very large scales. Our main tool for doing so is the Cosmic Microwave Background (CMB), and its physical properties. The study is based on the methodology suggested by \citet{callinHowCalculateCMB2006}.

This paper uses the Einstein equations and various results from statistical thermodynamics to derive numerical equations describing first a simple $\Lambda$CDM cosmology, and then seed this cosmology with initial perturbations, which theory tells us will be sourced from inflation. Evolution of these perturbations leads to observable anisotropies in the CMB, which will be reflected when we derive the CMB (and matter) power spectrum. The physical basis for doing this is supported by \citet{dodelsonModernCosmology2003}, and the specific methodology and formalism is described by \citet{wintherCosmologyIILecture2024}.

The main goal is thus to be able to predict the CMB (and matter) fluctuations via the power spectrum from first principles. By being able to match observational data, we prove that our initial assumptions are reasonable and can apply for the real universe, such as inflation. By removing the impact of inflation or altering various cosmological parameters (like removing dark matter) from our simulation, we can produce results irreconcilable with observational data, which indicates that our universe includes these effects - or that the theory of relativity is incorrect, which seems unlikely to say the least.

The process of deriving the equations and then numerical implementation is a great tool to learn about the intricacies of modern cosmology \citep{callinHowCalculateCMB2006}. 

This study is baselined on the common Lambda-CDM model \citep[chap. 1][sec. 1.6]{dodelsonModernCosmology2021}, with a {Friedmann}-{Robertson}-{Walker} metric for spacetime.
   
%-------------------------------------------------------------------
