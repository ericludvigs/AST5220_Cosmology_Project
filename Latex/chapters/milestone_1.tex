%\FloatBarrier
\section{Milestone I}\label{sec:milestone_1}
In this milestone, we establish the so-called background cosmology of our universe for the simulation. We focus on some rough supernova fitting to give an initial target for the spacetime metric and relative energy densities we will work with.

Citations: \citet{baumannLectureNotesCosmology2017}, \citet{dodelsonModernCosmology2003} and \citep{callinHowCalculateCMB2006, wintherCosmologyIILecture2024, huCompleteTreatmentCMB1998}

\subsection{Theory}
The scale factor $a$ is used as the main time-ish variable - $a$ is not a time but since time, size of the universe, and cosmic distances are all closely related it can fulfill the role of a time variable nonetheless, and describe the evolution of our simulated universe \citep{wintherCosmologyIILecture2024}. The scale factor is a dimensionless quantity.

For numerical work we use the variable $x \equiv \log a$, this is for numerical stability over large timescales with highly variable quantities. 

We also consider the cosmic time $t$, and in theory $a$ is dependent on $t$, $a(t)$, but in practice we derive the cosmic time $t$ from $x$ (and thus $a$) and not the other way around, see eq. \ref{eq:t_of_x}.

We define $a_0$ as the value of the scale factor today, such that $a_0 \equiv a(t_{\text{today}}) \equiv 1$. Note that $x_0 = 0$. In general, subscript $_0$ indicates the value of a parameter as measured in the current day.

Another related quantity is the redshift $z$, this is also $z_0 = 0$ in the present day. As observers we must by definition be at zero redshift, after all. Redshift is given by $1+z = a_0 / a(t) = 1/a$.

\newpage
Fiducial cosmology and initial parameter values taken from the Planck 2018 results \citep{collaborationPlanck2018Results2020}. Input parameters are listed at \ref{eq:input_params}.

\begin{equation}\label{eq:input_params}
\begin{aligned}
h &= 0.67, \\
T_{\rm CMB 0} &= 2.7255\,\unit{K}, \\
N_{\rm eff} &= 3.046, \\
\Omega_{\rm b 0} &= 0.05, \\
\Omega_{\rm CDM 0} &= 0.267,\\
\Omega_{k 0} &= 0, \\
\Omega_{\nu 0} &= N_{\rm eff}\cdot \frac{7}{8}\left(\frac{4}{11}\right)^{4/3}\Omega_{\gamma 0}, \\
\Omega_{\Lambda 0} &= 1 - (\Omega_{k 0}+\Omega_{b 0}+\Omega_{\rm CDM 0}+\Omega_{\gamma 0}+\Omega_{\nu 0}),\\
n_s &= 0.965, \\
A_s &= 2.1\cdot 10^{-9}, \\
Y_p &= 0.245, \\
z_{\rm reion} &= 8, \\
\Delta z_{\rm reion} &= 0.5, \\
z_{\rm He-reion} &= 3.5, \\
\Delta z_{\rm He-reion} &= 0.5.
\end{aligned}
\end{equation}

\subsubsection{Friedmann equation}
Rather than dealing with the full Einstein equations directly, it is possible to derive the Friedmann equation in order to describe the expansion of the universe, this is eq. \ref{eq:Friedmann} \citep{wintherCosmologyIILecture2024}.

\begin{equation}\label{eq:Friedmann}
\boxed{H = H_0 \sqrt{ \Omega_{M 0} a^{-3} + \Omega_{R 0} a^{-4} + \Omega_{k 0} a^{-2} + \Omega_{\Lambda 0}}}\,,
\end{equation}

where the $\Omega_{X}$ are density parameters describing relative density of their respective form of energy contributing to the expansion of the universe. Density parameters are dimensionless. Subscript $_0$ indicates a value for the universe of today, since we as observers are by definition at $x=z=0$.

$\Omega_{M} = (\Omega_{b}+\Omega_{\rm CDM})$ is a composite density parameter describing non-relativistic matter (baryons and cold dark matter), and $\Omega_{R} = (\Omega_{\gamma} + \Omega_{\nu})$ is a composite density term for radiation (photons and neutrinos).
$\Omega_{\Lambda}$ is the density parameter for dark energy.
$\Omega_{k}$ is a curvature term, and not properly an energy density. However, it contributes to the Friedmann equation as if it were a normal matter fluid with equation of state $\omega = -1/3$. This term prescribes negative curvature when $<0$, positive curvature when $>0$, and a spatially flat universe when $=0$. Our universe is observationally confirmed to be very close to flat \citep{bennettNineYearWilkinsonMicrowave2013}, so this term should be close to $0$. 

The unit of $H$ is slightly ambiguous, as in SI units both \unit{(km \per s)\per Mpc} and the simplified \unit{1 \per s} are used. The first way of writing the unit is more intuitive, as it relates to the change in velocity of distant galaxies based on their distance from the observer (us), aka Hubble's law. \unit{1 \per s} is technically correct but interpreting the Hubble parameter as a frequency is not helpful. For the Friedmann equation and derived quantities, we mostly skip this problem by keeping $H$ in "units of $H$", using the value of the Hubble parameter today ($H_0$) as a constant which gives the right units to any $H$ that pops up.

For numerical work, we calculate the constant $H_0$ by adding the right units to a dimensionless constant $h$ (eq. \ref{eq:little_h}), which is commonly used and reported in the literature \citep{crotonDamnYouLittle2013}. $h$ is one of the input observables for our numerical simulation, so we use the $0.67$ value reported by Planck 2018 \citep{collaborationPlanck2018Results2020}.

\begin{equation}\label{eq:little_h}
H_0 = 100 * h \enspace \unit{km.s^{-1}.Mpc^{-1}}
\end{equation}

\subsubsection{More stuff}

Paper with supernova fitting data \citet{betouleImprovedCosmologicalConstraints2014}.

Equation for critical density of the universe today \ref{eq:critical_density}

\begin{equation}\label{eq:critical_density}
\rho_{c0} \equiv \frac{3H_0^2}{8\pi G} \enspace \unit{kg.m^{-3}}
\end{equation}

\begin{equation}\label{eq:t_of_x}
t(x) = \int_0^a \frac{da}{aH} = \int_{-\infty}^x \frac{dx}{H(x)} \quad [\text{unit \unit{s}, can convert to \unit{\giga yr} etc.}]
\end{equation}

\subsection{Implementation details}
We run a Markov chain Monte Carlo simulation on a large number of parameter combinations, in order to test numerical fitting to real supernova data. This ensures our background parameters and simulation is sensible, before we add perturbations to simulate a dynamic universe development through time.

\subsection{Results}
See figs. \ref{fig:milestone_1_cosmic_vs_conformal_time}, \ref{fig:milestone_1_luminosity_distance}, \ref{fig:milestone_1_etaHp_over_c_of_x}, \ref{fig:milestone_1_Omega_i_of_x}, \ref{fig:milestone_1_supernovafitting_confidence_regions}. Overall these figures seem ok, except that the supernova fitting in fig. \ref{fig:milestone_1_luminosity_distance} misses systematically, especially at the beginning. There is probably some bug involved, since the shape seems correct.

Best-fit parameters are
Best h: 0.706857,
Best H0: 70.6857 km/s/Mpc,
Best OmegaM: 0.268287,
Best OmegaK: 1.84807e-05,
Best OmegaLambda: 0.731694519

Via the cosmic time, the calculated current age of the Universe is: $13.849 * 10^9$ yrs.
This is pretty close to 13.79 billion years, the current accepted estimate.

\begin{figure}[h!tbp]
\centering
\includegraphics[width=0.4\textwidth]{../Milestone 1/Plots/cosmic_vs_conformal_time.png}
\caption{Cosmic and conformal time plotted against the expansion of the universe. The cosmic time evolves almost linearly, as we'd expect.}
\label{fig:milestone_1_cosmic_vs_conformal_time}
\end{figure}

\begin{figure}[h!tbp]
\centering
\includegraphics[width=0.4\textwidth]{../Milestone 1/Plots/luminosity_distance.png}
\caption{Luminosity distance against redshift, with larger redshifts representing objects further away. This evolution is strongly related to the Hubble parameter and Hubble's law.}
\label{fig:milestone_1_luminosity_distance}
\end{figure}

\begin{figure}[h!tbp]
\centering
\includegraphics[width=0.4\textwidth]{../Milestone 1/Plots/etaHp_over_c_of_x.png}
\caption{A plot of the conformal Hubble factor $\mathcal{H} = aH$, scaled against the analytical solution of the Friedmann equation in the radiation dominated era ($\eta \simeq \frac{c}{\mathcal{H}}$). This should converge to $1$ the more radiation dominated the universe is, and shows the correct exponential decay compared to the decay of relativistic matter seen in fig. \ref{fig:milestone_1_Omega_i_of_x}, if not the right value at the beginning of the plot.}
\label{fig:milestone_1_etaHp_over_c_of_x}
\end{figure}

\begin{figure}[h!tbp]
\centering
\includegraphics[width=0.4\textwidth]{../Milestone 1/Plots/Omega_i_of_x.png}
\caption{The evolution of the various energy components from the Friedmann equation. We clearly see the three eras of radiation-domination, matter-domination, and dark energy beginning to dominate right around the present day, as expected. For reference, the sum of the components is calculated at every point and plotted at the top - this exeeding 100\% energy at any point would be very problematic, but it stays perfectly at 1.}
\label{fig:milestone_1_Omega_i_of_x}
\end{figure}

\begin{figure}[h!tbp]
\centering
\includegraphics[width=0.4\textwidth]{../Milestone 1/Plots/supernovafitting_confidence_regions.png}
\caption{A scatter plot of the various parameter combinations that were attempted for fitting by the MCMC. Parameters outside the blue area are extremely unlikely. For reference, the combinations corresponding to a spatially flat universe are the ones on the dotted black line. Since we know observationally that our universe is close to flat, a dark energy content of about 60-80\% is likely, with a corresponding matter density of about 20-40\%.}
\label{fig:milestone_1_supernovafitting_confidence_regions}
\end{figure}
